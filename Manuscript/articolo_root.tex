\documentclass[a4paper]{siamonline220329}

\usepackage{damacros}

% PACKAGES added by TB
% PACKAGES FOR ALGORITHMS (PSEUDO-CODE)
\usepackage{algorithm}
\usepackage{algorithmic}

% Sets running headers as well as PDF title and authors
\headers{Coupling intra-cellular and multi-cellular dynamics in spatially-extended models of root-hair initiation}{D. Avitabile, S. Perotto, N. Ferro, T. Babini}

% Title. If the supplement option is on, then "Supplementary Material"
% is automatically inserted before the title.
\title{Coupling intra-cellular and multi-cellular dynamics in spatially-extended models of root-hair initiation}

% Authors: full names plus addresses.
\author{%
  Daniele Avitabile%
  \thanks{%
    Vrije Universiteit Amsterdam,
    Department of Mathematics,
    Faculteit der Exacte Wetenschappen,
    De Boelelaan 1081a,
    1081 HV Amsterdam, The Netherlands.
  \protect\\
    Inria Sophia Antipolis M\'editerran\'ee Research Centre,
    MathNeuro Team,
    2004 route des Lucioles-Boîte Postale 93 06902,
    Sophia Antipolis, Cedex, France.
  \protect\\
    (\email{d.avitabile@vu.nl}, \url{writemywebpage}).
  }
 %  .... mancano altri
  % \and
%   Paul T. Frank \thanks{Department of Applied Mathematics, Fictional University, Boise, ID
% (\email{ptfrank@fictional.edu}, \email{jesmith@fictional.edu}).}
% \and Jane E. Smith\footnotemark[3]
}

\begin{document}

\maketitle

\begin{abstract}
This thesis deals with novel models and numerical approximations of
spatially-extended multi-cellular models of Rho Of Plants (ROPs), that is, a family
of proteins responsible for root-hair initiation in the plant cell Arabidopsis
thaliana. The study of this dynamical system is of great relevance in the so-called
agriculture 4.0, since it is instrumental to optimise plant uptake. In particular,
ascertaining how intra-cellular protein distributions and extra-cellular coupling
influence root-hair initiation is a challenging but pressing problem.

Current studies have focussed on two separate model types: on the one hand, ROPs dynamics is studied in single-cell models, which resolve patterns at sub cellular level; on the other, multi- cellular models with realistic geometries neglect intra-cellular patterning. In this thesis we make progress on coupling these two
model descriptions.

We initially focus on a well-established single-cell, nonlinear reaction-diffusion model, here approximated for the first time with a finite-element scheme. In addition, we present a new model which couples multiple cells through ROP flux at the interface. We present numerical evidence that such coupling has a bearing on the patterns supported by the model. It is shown that, under variations of auxin gradients, the model robustly forms ROP hotspots from ROP stripes, and that spots are later advected downstream.

Finally, we consider a novel model in which the auxin dynamics are not prescribed, but derive from the interaction between this hormone and other membranals proteins (PIN). We show that self-sustained auxin oscillations influence ROP intracellular patterning.
\end{abstract}

\da[inline]{Daniele's rewrite begins}

\section{Intro}

\section{Single cell model}\label{sec:singleCell} 

We shall denote by $\Omega \subset \RSet^3$ the cell interior, which is assumed to be
open and bounded, and by $\Gamma = \partial \Omega$ its boundary. \da[]{technical
assumption on smoothness of $\Omega$} Importantly, we will model the evolution of
reactants within the cell on timescales that are short compared with timescales for
cellular growth and division, therefore $\Omega$ and $\Gamma$ are assumed to be
time-independent.

As an exemplary setup for intracellular dynamics, we consider a model of root-hair
initiation which describes the evolution of two concentrations, namely $u(x,t)$, and
$v(x,t)$, describing the density of active and inactive Rho-Of-Plants (ROPs), at
position $x \in \Omega$ and $t \in [0,T] \subset \RSet$, respectively. \da[]{Cite} 

In addition, to the unknown fields $u,v \colon \overline{\Omega} \times [0,T] \to
\RSet_{ \geq  0}$, the model involves a prescribed, exhogenous, spatio-temporal auxin
field, $\alpha \colon \overline{\Omega} \times [0,T] \to \RSet_{ \geq  0}$. As a
first modelling assumption, we will assume that the auxin dynamics be assigned, and
decoupled from the ROP dynamic. As we will see later, this hypothesis can be relaxed.

Following XXX\da[]{Cite}, we model the evolution of active and inactive ROPs using a
heterogeneous reaction-diffusion system which, in dimensionless form, is written
as
  \begin{align}
    & \partial_t u = d_u \Delta u + f(u,v,\alpha), 
      && \partial_t v = d_v \Delta v + g(u,v,\alpha), 
      && \text{on $\Omega\times (0,T]$,} \label{eq:RDEs}\\
    & d_u \nabla u \cdot n = \psi_u, 
      && d_v \nabla v \cdot n = \psi_v, 
      && \text{on $\Gamma \times [0,T]$,} \label{eq:BCs}\\
    & u = u_0,  && v = v_0, && \text{on $\Omega \times \{0\}$.} \label{eq:ICs}
  \end{align}
where $d_u, d_v \in \RSet_{ \geq 0}$ are diffusion constants, $f$ and $g$ \da[]{Maybe
$f_u,f_v$?} denote the auxin-dependent, nonlinear
reaction terms
\[
  \begin{aligned}
   & f(u,v,\alpha) = \alpha u^2 v - \beta_u u + \gamma v, \\
   & g(u,v,\alpha) = -\alpha u^2 v + \beta_v u - \gamma v + \delta,
  \end{aligned}
\]
and where $\beta_u, \beta_v, \gamma, \delta$ are scalar control parameters. ROPs
activate (resp. inactivate) at a rate proportional to the local auxin concentration,
and with a cubic polynomial nonlinearity; in passing, we note that, while the
linear $v$-dependent production and depletion terms are identical and with opposite signs,
the lineear $u$-dependent production and depletion rate differ, as $\beta_u \neq
\beta_v$; further, there is a constant production rate term in the function $g$,
which is absent in $f$. \da[]{Must be better explained after talking to Teresa.}

The evolution equations \cref{eq:RDEs} are complemented by heterogeneous Neumann
boundary conditions with fluxes $\psi_u$, $\psi_v$, and initial
conditions $u_0$, $v_0$ in \cref{eq:BCs} and \cref{eq:ICs}, respectively.

\da[inline]{Daniele's rewrite ends}

\section{Introduction (Daniele+Teresa+Simona)}

mettere il modello Capitolo 4 (solo equazioni) +  posizionamento in
letteratura + novelty paper

% from chapter 4 physical model sec.

the dynamics of auxin and carriers proteins PIN in each cell $\Omega_i$ with $i \in \{ 1, ..., N \}$ can be written as the following system of coupled ordinary differential equations:
\begin{equation}\begin{aligned}
\begin{cases}
  {\displaystyle d a_i\over\displaystyle d t} & = {\displaystyle 1 \over \displaystyle V_i} \sum_{j=1}^{N} A_{ij} \Phi_{ij} +k - \delta a_i \\[8pt]
  {\displaystyle d \Tilde{P}_{ij}\over\displaystyle d t} & = h\left(\Phi_{ij}\right) + \rho_0 - \mu \Tilde{P}_{ij},
\end{cases}
\end{aligned} \end{equation}

% ...

The original system of equations can be rescaled and simplified. In particular, under the assumptions of cells having same volume $V =V_i$ and exchange surface areas $A = A_{ij}$, rescaling properly the diffusion coefficient $D_a$ and variables $P_{ij}$, a new system is obtained. Thus, the final system of equations we work on is:
\begin{equation}\label{eq:Sys_auxPIN}\begin{aligned}
\begin{cases}
  {\displaystyle d a_i\over\displaystyle d t} & = \sum_{j \in \mathcal{N}_i} \displaystyle \Phi_{ji} +k - \delta a_i \\[8pt]
  {\displaystyle d P_{ij}\over\displaystyle d t} & = h\left(\Phi_{ij}\right) - \mu P_{ij},
\end{cases}
\end{aligned} \end{equation}

% ... from chapter 2 sec physical model
The dimentional reaction-diffusion model summarizing binding process, autocatalytic activation and catalysis of ROPs proteins described in Section \ref{sec:intromodel} is formulated as follows:
% As mentioned before, the dimentional reaction-diffusion model is:
\begin{equation} \label{eq:FM}\begin{aligned}
\left\lbrace
\begin{matrix}
\partial_t u = & D_1 \Delta_s u + k_{20} \alpha(x,y)u^2 v - \left(c+r\right) u + k_1 v  & \ \text{in} \ \Omega\\
    \partial_t v = & D_2 \Delta_s v - k_{20} \alpha(x,y) u^2 v - k_1 v + c \ u + b   & \ \text{in} \ \Omega.
\end{matrix}
\right.
\end{aligned}\end{equation}

% ...
The RD system is rewritten to explicit its mathematical formulation and methods in order to solve it in the most general way, including both the case of the RD system with the original parameters and the RD system with the rescaled ones.

As a consequence, we define a unified system of equations comprehensive of both \eqref{eq:FM} and \eqref{eq:adim} systems:
\begin{equation} \label{eq:final}
\left\lbrace
\begin{matrix}
\partial_t u = & \Tilde{D_1} \Delta_s u + \Tilde{a_1} u + \Tilde{b_1} v + \Tilde{c_1} u^2 v \ \ \text{in} \ \Omega\\
\partial_t v = & \Tilde{D_2} \Delta_s v + \Tilde{a_2} v + \Tilde{b_2} u + \Tilde{c_2} u^2 v + f_2 \ \ \text{in} \ \Omega,
\end{matrix}
\right.
\end{equation}
with tilde parameters defined differently in the two cases under study

\section{A starting modeling  (Teresa+Daniele)}

Modello di Capitolo 3

\textbf{Physical model - Sec.3.1}
We consider the root-hair cell projection onto a 2D rectangular domain, neglecting axial dimension.
% as in Chapter \ref{cap:2}
A system of four cells is schematically presented in Figure \ref{fig:2cell}. We can see that each cell has longitudinal and transverse boundaries in common with close cells.
We recall the single cellular model, namely:
\begin{equation} \label{eq:singModel}
\left\lbrace
\begin{matrix}
  \begin{aligned}
    & \partial_t u = \Tilde{D_1} \Delta_s u + \Tilde{a_1} u + \Tilde{b_1} v + \Tilde{c_1} u^2 v & \ \text{in} \ \Omega\\
    & \partial_t v = \Tilde{D_2} \Delta_s v + \Tilde{a_2} v + \Tilde{b_2} u + \Tilde{c_2} u^2 v + f_2 & \ \text{in} \ \Omega \\
    & \Tilde{D_1} \nabla_s u \cdot \mathbf{n} = 0 & \ \text{on} \ \partial \Omega \\
    & \Tilde{D_2} \nabla_s v \cdot \mathbf{n} = 0 & \ \text{on} \ \partial \Omega.
  \end{aligned}
\end{matrix}
\right.
\end{equation}
No-flux on $\partial \Omega$, namely Neumann homogeneous boundary conditions, characterizes the system behaviour along the cell boundary.
% prima spieghiamo il significato fisico, poi come rappresentarlo matematicamente
In the multi-cellular model, communication between cells is represented by allowed flux of ROPs, active and inactive, through localized channels along boundaries between neighboring cells.

We define as neighbor of cell $\Omega_i$ the set of cells with index in $\mathcal{N}_i = \{ j : \partial \Omega_j  \cap \partial \Omega_i \neq \emptyset \}$. The flux of concentration of active and inactive ROPs $(u_i, v_i)$ is proportional to the difference of concentration $(u_j, v_j)$ in neighbouring cells for $j \ \in \ \mathcal{N}_i$.

We formulate the new model still focusing on one single cell domain $\Omega_i$, taking into account the new flux generated from the discrepancy of concentrations with the neighboring cells. The new flux results in adding a non-homogeneous Neumann boundary condition on the common interfaces, as follows:
\begin{equation} \label{eq:pluriModel}
\left\lbrace
\begin{matrix}
  \begin{aligned}
    & \partial_t u_i = \Tilde{D_1} \Delta_s u_i + \Tilde{a_1} u_i + \Tilde{b_1} v_i + \Tilde{c_1} (u_i)^2 v_i & \ \text{in} \ \Omega_i\\[6pt]
    & \partial_t v_i = \Tilde{D_2} \Delta_s v_i + \Tilde{a_2} v_i + \Tilde{b_2} u_i + \Tilde{c_2} (u_i)^2 v_i + f_2 & \ \text{in} \ \Omega_i \\[6pt]
    & \Tilde{D_1} \nabla_s u_i \cdot \mathbf{n} = 0 & \ on \ \partial \Omega_i \backslash \cup_{j \in \mathcal{N}_i} \Gamma_{j,i} \\[6pt]
    & \Tilde{D_2} \nabla_s v_i \cdot \mathbf{n} = 0 & \ on \ \partial \Omega_i \backslash \cup_{j \in \mathcal{N}_i} \Gamma_{j,i} \\[6pt]
    & \Tilde{D_1} \nabla_s u_i \cdot \mathbf{n} = \beta_{uRR} \ \alpha_{uRR} \left(u_j - u_i \right) & \ on \ \Gamma_{j,i} \ \forall j \in  \mathcal{N}_i \\[6pt]
    & \Tilde{D_2} \nabla_s v_i \cdot \mathbf{n} = \beta_{vRR} \ \alpha_{vRR} \left(v_j - v_i \right) & \ on \ \Gamma_{j,i}\ \forall j \in  \mathcal{N}_i ,
  \end{aligned}
\end{matrix}
\right.
\end{equation}
where we define as $(u_i, v_i)$ the concentrations of active and inactive ROPs restricted to cell $\Omega_i$: $(u_i, v_i): \Omega_i \times \left(0, T_{max} \right) \longrightarrow \RSet^2$ and $\Gamma_{j,i}$ represents the common side between cell $\Omega_i$ and cell $\Omega_j \ \in \mathcal{N}_i $, therefore defined as: $\Gamma_{j,i} = \partial \Omega_i \cap \partial \Omega_j$.

Each of the neighboring cells follows the same model for hair formation, meaning that system in \eqref{eq:pluriModel} holds $\forall \ i$ cells composing the pluricellular system. As a consequence, the newly defined boundary conditions is coupled with the solutions $(u_j, v_j)$ with $j \ \in \mathcal{N}_i$. Therefore, the pluricellular system requires a proper iterative method for setting correctly boundary conditions depending on solutions in the neighboring cells.

Not communicating with other RH cells boundaries have as before no-flux. The new boundary conditions are characterized by a function and a coefficient for both active active ROPs $u$ and inactive ROPs $v$, having the same meaning:
\begin{itemize}
  \item $\beta_{u/v RR} \ [\frac{1}{\mu m^2}]$ are indicator functions defined on boundaries of cells, equal to $1$ where the communicating channels are open and $0$ where  no-flux is assumed;
  \item $\alpha_{u/v RR} \ [\frac{1}{\mu m}]$ are transport efficiency coefficients, representing a sort of flux quantity allowed through channels.
  \end{itemize}
These channel parameters aim at representing the average active transport along the sides of confining cells, set equal to the flux of proteins from one cell to the neighbouring ones.

We have no physical insight on previously cited functions modeling open channels for ROPs. A whole set of simulations for the proper tuning of parameters is required, in order to find a sufficiently plausible setting of the system.

\section{A new dd-wise coupling approach (Simona+Daniele+Nicola+Teresa)}

DD sul modello del Capitolo 3 + parte discreta

\textbf{Numerical treatment - Sec.3.2}
% brutto mettere gli stessi titoli?
% -> prima questione: trattarlo con uno schema iterativo simile robin robin algorithm of domain decomposition, in modo da accoppiare le soluzioni; quindi descrivere bene lo schema (dalla strong formulation); già partendo dalo schema semi implicit nel tempo
The communication between cells requires a proper iterative algorithm in order to deal with the mutual interplay between confining cells.

Every subdomain $\Omega_i$ of the pluricellular system $\Omega$ represents the single cell and the original system of equations in \eqref{eq:final} is solved in $\Omega_i$ for all $i = 1, ..., N$. We solve such systems by means of the semi-implicit method described in Section \ref{sec:SI method}. Let us consider the weak formulation restricted to $\Omega_i$, defining the functional space $V_i = \{ w_i \in \ H^1\left(\Omega_i\right)\}$, the finite element subspace $V_{i,h} \subset V_i $ and the time interval discretization used in Section \ref{sec:SI method}. In particular, we divide the time interval $\left[0, T_{max}\right]$ in $N_{max}$ time steps such that $t^n = n \Delta t$ with $\Delta t = T_{max} / N_{max}  $. We rewrite the full discretized formulation, identifying $u_{i,h}$ with $u_h|_{\Omega_i}$, as:

given the initial state $(u_{i,h}^0, v_{i,h}^0) $, find $(u_{i,h}^{n+1}, v_{i,h}^{n+1}) \ \in V_{i,h} \times V_{i,h}$ such that
\begin{equation} \label{eq:fullGalerkin}
\left\lbrace
\begin{matrix}
\begin{aligned}
  a_{i,u}(u_{i,h}^{n+1}, w_{i,h}) + b_{i,u}(v_{i,h}^{n+1}, w_{i,h}) + c_{i,u}(v_{i,h}^{n+1}, w_{i,h}) = f_{i,u}(w_{i,h}) \ \forall \ w_{i,h} \ \in V_{i,h} \\[6pt]
 a_{i,v}(v_{i,h}^{n+1}, w_{i,h}) + b_{i,v}(u_{i,h}^{n+1}, w_{i,h}) + c_{i,v}(v_{i,h}^{n+1}, w_{i,h}) = f_{i,v}(w_{i,h}) \ \forall \ w_{i,h} \ \in V_{i,h},
\end{aligned}
\end{matrix}
\right.
\end{equation}
$\forall n = 0, ... N_{max}$, where
\begin{subequations} \label{eq:Gvarfmono}
\begin{align}
    a_{i,u}(u_{i,h}^{n+1}, w_{i,h}) = & \int_{\Omega_i} \left( \frac{1}{\Delta t} u_{i,h}^{n+1} w_{i,h} + \Tilde{D}_1 \nabla_s u_{i,h}^{n+1} \cdot w_{i,h} - \Tilde{a}_1 u_i^{n+1} w_{i,h} \right) \label{Gmono:au}\\ - & \int_{\partial \Omega_i}\left(\Tilde{D}_1 \nabla_s u_{i,h}^{n+1} \cdot \mathbf{n} w_{i,h}\right)  \nonumber\\
    b_{i,u}(v_{i,h}^{n+1}, w_{i,h}) = & \int_{\Omega_i} \left(- \Tilde{b}_1 v_{i,h} w_{i,h} \right)  \label{Gmono:bu} \\
    c_{i,u}(v_{i,h}^{n+1}, w_{i,h}) = & \int_{\Omega_i} \left(- \Tilde{c}_1 (u_{i,h}^{n})^2  v_{i,h}^{n+1} w_{i,h} \right) \label{Gmono:cu} \\[6pt]
    a_{i,v}(v_{i,h}^{n+1}, w_{i,h}) = & \int_{\Omega_i} \left(\frac{1}{\Delta t} v_{i,h}^{n+1} w_{i,h} + \Tilde{D}_2 \nabla_s v_{i,h}^{n+1} \cdot w_{i,h} - \Tilde{a}_2 v_i^{n+1} w_{i,h} \right) \label{Gmono:av} \\ - &  \int_{\partial \Omega_i} \left( \Tilde{D}_1 \nabla_s u_{i,h}^{n+1} \cdot \mathbf{n} w_{i,h} \right) \nonumber\\
    b_{i,v}(v_{i,h}^{n+1}, w_{i,h}) = &\int_{\Omega_i} \left( - \Tilde{b}_2 u_{i,h} w_{i,h} \right) \label{Gmono:bv}\\
    c_{i,v}(v_{i,h}^{n+1}, w_{i,h}) =& \int_{\Omega_i} \left( - \Tilde{c}_2 (u_{i,h}^{n})^2  v_{i,h}^{n+1} w_{i,h} \right) \label{Gmono:cv}\\[6pt]
    f_{i,u}(w_{i,h}) = & \int_{\Omega_i} \left( \frac{1}{\Delta t} u_{i,h}^n \ w_{i,h} \right) \label{Gmono:fu}\\
    f_{i,v}(w_{i,h}) = & \int_{\Omega} \left( \frac{1}{\Delta t} v_{i,h}^n \ w_{i,h} + f_2 w_{i,h} \right). \label{Gmono:fv}
\end{align}
\end{subequations}

The introduction of different boundary conditions will require to modify the bilinear forms \eqref{Gmono:au} and \eqref{Gmono:av} and to add contributions in the right hand sides \eqref{Gmono:fu} and \eqref{Gmono:fv}.

To this aim, we synthetically rewrite the model problem \eqref{eq:fullGalerkin}, assuming generic boundary conditions, through a linear operator $\mathcal{L}$ in the following way:

Given the initial state $(u_i^0, v_i^0)$, find $(u_i^{n+1}, v_i^{n+1}) \ \in \Omega_i$ such that:
\begin{equation}\label{eq:modelpb}
% \begin{cases}
\mathcal{L}^n (u_i^{n+1}, v_i^{n+1}) = \mathbf{f}^n \ \text{in} \ \Omega_i
% \Tilde{D_1} \nabla_s u_i^{n+1} \cdot \mathbf{n} = 0 \ \text{on} \ \partial \Omega_i \\
% \Tilde{D_2} \nabla_s v_i^{n+1} \cdot \mathbf{n} = 0 \ \text{on} \ \partial \Omega_i
% \end{cases}
\end{equation}
$\forall n = 0, ... N_{max}$.

\textbf{A new iterative modeling algorithm - Sec 3.2.3 }
The model that we propose to make cells communicate can be regarded as a simplification of the classical domain decomposition scheme with Robin boundary conditions. We start for simplicity from a two cells problem and rewrite the common interface boundary conditions in \eqref{eq:RR} to recover the modelled open channels in \eqref{eq:pluriModel}. In the spirit of a block-Gauss-Seidel algorithm, we solve in sequence:
\begin{equation} \label{eq:RR_final}
\begin{aligned}
& \begin{cases}
\mathcal{L}^n (u_1^{k+1}, v_1^{k+1}) = \mathbf{f}^n \ \text{in} \ \Omega_1 \\
\Tilde{D_1} \nabla_s u_1^{k+1} \cdot \mathbf{n} = 0 \ \text{on} \ \partial \Omega_1 \setminus \Gamma \\
\Tilde{D_1} \displaystyle{\partial u_1^{k+1}\over\partial \mathbf{n}} = \alpha_{uRR} u_2^{k} - \alpha_{uRR} u_1^{k+1} \ \text{on} \ \Gamma \\
\Tilde{D_2} \nabla_s v_1^{k+1} \cdot \mathbf{n} = 0 \ \text{on} \ \partial \Omega_1 \setminus \Gamma \\
\Tilde{D_2} \displaystyle{\partial v_1^{k+1}\over\partial \mathbf{n}} = \alpha_{vRR} v_2^{k} -\alpha_{vRR} v_1^{k+1} \ \text{on} \ \Gamma
\end{cases}
\\[6pt]
& \begin{cases}
\mathcal{L}^n (u_2^{k+1}, v_2^{k+1}) = \mathbf{f}^n \ \text{in} \ \Omega_2 \\
\Tilde{D_1} \nabla_s u_2^{k+1} \cdot \mathbf{n} = 0 \ \text{on} \ \partial \Omega_2 \setminus \Gamma \\
\Tilde{D_1} \displaystyle{\partial u_2^{k+1}\over\partial \mathbf{n}} = \alpha_{uRR} u_1^{k+1} - \alpha_{uRR} u_2^{k+1} \ \text{on} \ \Gamma \\
\Tilde{D_2} \nabla_s v_2^{k+1} \cdot \mathbf{n} = 0 \ \text{on} \ \partial \Omega_2 \setminus \Gamma\\
\Tilde{D_2} \displaystyle{\partial v_2^{k+1}\over\partial \mathbf{n}} = \alpha_{vRR} v_1^{k+1} -\alpha_{vRR} v_2^{k+1} \ \text{on} \ \Gamma.
\end{cases}
\end{aligned}\end{equation}
The flux imposed depends on the difference of the neighbouring solutions. As a consequence, we are imposing a not necessarily null Neumann boundary condition. Equation \eqref{eq:RR_final} defines a RR iterative method applied to two cells using proper parameters $\beta_{u/vRR}$ and $\alpha_{u/v RR}$ from the model formulated in Section \ref{sec:PluriMod}:

starting from $(u_2^{k = 0}, v_2^{k = 0}) = (u_2^n, v_2^n)$, find $(u_1^{k+1}, v_1^{k+1}) \ \in V_{1}$ and $(u_2^{k+1}, v_2^{k+1}) \ \in V_{2}$:
\begin{equation}\label{eq::RRmod}
\begin{aligned}
& \begin{cases}
\mathcal{L}^n (u_1^{k+1}, v_1^{k+1}) = \mathbf{f}^n \ \text{in} \ \Omega_1 \\
\Tilde{D_1} \nabla_s u_1^{k+1} \cdot \mathbf{n} = 0 \ \text{on} \ \partial \Omega_1 \setminus \Gamma \\
\Tilde{D_1} \displaystyle{\partial u_1^{k+1}\over\partial \mathbf{n}} = \beta_{uRR} \alpha_{uRR} \left( u_2^{k} - u_1^{k+1} \right) \ \text{on} \ \Gamma \\
\Tilde{D_2} \nabla_s v_1^{k+1} \cdot \mathbf{n} = 0 \ \text{on} \ \partial \Omega_1 \setminus \Gamma \\
\Tilde{D_2} \displaystyle{\partial v_1^{k+1}\over\partial \mathbf{n}} = \beta_{vRR} \alpha_{vRR} \left( v_2^{k} - v_1^{k+1} \right) \ \text{on} \ \Gamma
\end{cases}
\\[6pt]
& \begin{cases}
\mathcal{L}^n (u_2^{k+1}, v_2^{k+1}) = \mathbf{f}^n \ \text{in} \ \Omega_2 \\
\Tilde{D_1} \nabla_s u_2^{k+1} \cdot \mathbf{n} = 0 \ \text{on} \ \partial \Omega_2 \\
\Tilde{D_1} \displaystyle{\partial u_2^{k+1}\over\partial \mathbf{n}} = \beta_{uRR} \alpha_{uRR} \left( u_1^{k+1} - u_2^{k+1} \right) \ \text{on} \ \Gamma \\
\Tilde{D_2} \nabla_s v_2^{k+1} \cdot \mathbf{n} = 0 \ \text{on} \ \partial \Omega_2 \setminus \Gamma\\
\Tilde{D_2} \displaystyle{\partial v_2^{k+1}\over\partial \mathbf{n}} = \beta_{vRR} \alpha_{vRR} \left( v_1^{k+1} - v_2^{k+1} \right) \ \text{on} \ \Gamma.
\end{cases}
\end{aligned}\end{equation}
for $k \geq 0$ up to convergence.

We remark that in \eqref{eq::RRmod} the coefficients $\beta_{u/vRR}$ and $\alpha_{u/v RR}$ have physical meaning since they come from the model \eqref{eq:pluriModel}. This is in contrast with the model and method presented in Section \ref{sec:RRclassic}, where the Robin coefficients are arbitrary.

Let $V_{i,h}$ denote the finite dimensional subspace of $H^1\left(\Omega_i\right)$, wih $\Omega_i$ being the sub-domain of the pluricellular system $\Omega$ corresponding to cell. We find solutions $\left(u_{h}^{n+1}, v_{h}^{n+1}\right)|_{\Omega_i}$ identified with $\left(u_{i,h}, v_{i,h}\right) \ \in V_{i,h}$ for each time step $t^{n+1}$, solving up to convergence the iteration step, whose Galerkin formulation is:
\begin{equation}\begin{aligned}
    a_{i,u}^{RR}(u_{i,h}^{k+1}, w_{i,h}) + b_{i,u}(v_{i,h}^{k+1}, w_{i,h}) + c_{i,u}^n(v_{i,h}^{k+1}, w_{i,h}) = f_{i,u}^{RR}(w_{i,h}) \ \forall w_{i,h} \in V_{i,h} \\
    a_{i,v}^{RR}(v_{i,h}^{k+1}, w_{i,h}) + b_{i,v}(u_{i,h}^{k+1}, w_{i,h}) + c_{i,v}^n(v_{i,h}^{k+1}, w_{i,h}) = f_{i,v}^{RR}(w_{i,h}) \ \forall w_{i,h} \in V_{i,h}.
\end{aligned}\end{equation}

The bilinear forms used are equal to \eqref{eq:Gvarfmono} - \eqref{eq:au&avRR} for classic Robin-Robin algorithm. The only difference is in the right hand side \eqref{eq:fu&fvRR} in which has been neglected the weak normal derivative of the neighbour solutions, as follows:
\begin{align}
 f_{i,u}^{RR}(w_{i,h}) & = f_{i,u}(w_{i,h}) + \int_{\Gamma} \left(\beta_{uRR} \alpha_{uRR} \mathcal{I}_{i,j} u_{j,h}^{k} \mathcal{I}_{i,j}w_{j,h} \right), \\
f_{i,v}^{RR}(w_{i,h}) & = f_{i,v}(w_{i,h}) + \int_{\Gamma} \left(\beta_{vRR} \alpha_{vRR} \mathcal{I}_{i,j} v_{j,h}^{k} \mathcal{I}_{i,j} w_{j,h} \right).
\end{align}

Consequently, the algebraic formulation of the new iterative method used is formulated similarly as in \eqref{eq:LinSysRR}, with time-dependent block matrix that need to be reassembled at each time-step. The right-hand sides depend on the previous solution found for the neighbouring cells and their contributions need to be interpolated by means of a interpolation matrix as in Robin-Robin classic method. We here explicit the whole iterative method for a two cells composed system.

Starting from initial guess given by the previous time-step solution  $\begin{bmatrix} \mathbf{U}_2^{0} \\ \mathbf{V}_2^{0} \end{bmatrix} = \begin{bmatrix} \mathbf{U}_2^{n} \\ \mathbf{V}_2^{n} \end{bmatrix}$, solve problem for $i = 1$ to find $\begin{bmatrix} \mathbf{U}_1^{k+1} \\ \mathbf{V}_1^{k+1} \end{bmatrix}$:
\begin{equation*}
\begin{aligned}
 \begin{bmatrix}
    A_u^1 & B_u^1 + C_u^1\left(  \mathbf{U}_1^n\right) \\
    B_v^1 & A_v^1 + C_v^1\left(  \mathbf{U}_1^n\right)
    \end{bmatrix} \begin{bmatrix}
    \mathbf{U}_1^{k+1} \\ \mathbf{V}_1^{k+1} \end{bmatrix} = \begin{bmatrix} F^1_u \left(\mathbf{U}_2^k\right) \\ F^1_v \left(\mathbf{V}_2^k\right)
    \end{bmatrix}
        \end{aligned}
\end{equation*}
and then solve problem for $ i = 2$ to find $\begin{bmatrix} \mathbf{U}_2^{k+1} \\ \mathbf{V}_2^{k+1} \end{bmatrix}$:
\begin{equation*}
    \begin{aligned}
\begin{bmatrix}
    A_u^2 & B_u^2 + C_u^2\left(  \mathbf{U}_2^n\right) \\
    B_v^2 & A_v^2 + C_v^2\left(  \mathbf{U}_2^n\right)
    \end{bmatrix} \begin{bmatrix}
    \mathbf{U}_2^{k+1} \\ \mathbf{V}_2^{k+1} \end{bmatrix} = \begin{bmatrix} F^2_u \left(\mathbf{U}_1^k\right) \\ F^2_v \left(\mathbf{V}_2^k\right)
    \end{bmatrix}
\end{aligned}\end{equation*}
for $k \geq 0$ up to convergence.

Iterations end when the normalized residual of consecutive computed solutions is smaller than a proper tolerance or when a maximum number of iterations is performed and we update the new solution as:
 $$\begin{bmatrix} \mathbf{U}_1^{n+1} \\ \mathbf{V}_1^{n+1} \end{bmatrix} = \begin{bmatrix} \mathbf{U}_1^{k+1} \\ \mathbf{V}_1^{k+1} \end{bmatrix}, \ \ \ \begin{bmatrix} \mathbf{U}_2^{n+1} \\ \mathbf{V}_2^{n+1} \end{bmatrix} = \begin{bmatrix} \mathbf{U}_2^{k+1} \\ \mathbf{V}_2^{k+1} \end{bmatrix}$$

Matrices and vectors used are defined in the following way:
\begin{equation}
    \begin{aligned}
    & \left[ A_u^i\right]_{j,l} & = a_{i,u}^{RR}(\phi_l, \phi_j), \ \ \ \left[ A_v^i\right]_{j,l} & = a_{i,v}^{RR}(\phi_l, \phi_j) \\
    & \left[ B_u^i\right]_{j,l} & = b_{i,u}(\phi_l, \phi_j), \ \ \ \left[ B_v^i\right]_{j,l} & = b_{i,v}(\phi_l, \phi_j)\\
    & \left[ C_u^i\right]_{j,l} & = c_{i,u}(\phi_l, \phi_j),\ \ \ \left[ C_v^i\right]_{j,l} & = c_{i,v}(\phi_l, \phi_j)\\
    & \left[F_u^i\right]_{j} & = f_{i,u}^{RR}(\phi_j), \ \ \
    \left[F_v^i\right]_{j} & = f_{i,v}^{RR}(\phi_j),
    \end{aligned}
\end{equation}
being $\{\phi_l\}_{l = 1}^{N_h}$ the functional basis of $V_{i,h}$ finite dimensional space defined on each cell $\Omega_i$ with $i =1,2$.

A sketch of the procedure to be adopted to deal with a generic N cells pluricellular system using Robin-Robin modifed algorithm is schematically given in Algorithm \ref{alg:RRmod}; $r_i$ are different coefficients characterizing initial state of concentrations, necessary for having flux between communicating cells. For physical reasons, we choose same initial guesses in the direction of the auxin gradient.

We have implemented a solver for a system of four cells.

As expressed in \eqref{eq::RRmod}, the iterative procedure is formulated such that the pluricellular domain is solved sequentially, in the sense that the boundary conditions characterizing sub-domain $i$, depending on sub-domain solutions of $j \in \mathcal{N}_i$, are computed using the newly updated solutions. In view of a parallel implementation, the method can be reformulated such that the new boundary conditions are a function of the previous iteration solution.

\begin{algorithm}[t]
    \caption{Pluricellular system solver procedure: RR}
    \label{alg:RRmod}
    Given $N \geq 1$ cells, $r_i$
    \begin{algorithmic}[1]
    \STATE Initialization: $\forall i = 1, ..., N$
    \STATE \verb|[U0i, V0i]| $\gets [r_i u_0, r_i v_0]$
    \STATE \verb|[Uiprec, Viprec]| $\gets$  \verb|[U0i, V0i]|
    \WHILE{$t < T_{max}$}
    \STATE{\verb|assemble| matrix for $\forall i = 1,..., N$}
    \FOR{$iter < Niter$}
    \STATE{$\forall i =1, ..., N$}
    \STATE{compute BC contribute from $j \in \mathcal{N}_i$}
    \STATE{\verb|interpolate| on $i$}
    \STATE{update \verb|rhs|}
    \STATE{\verb|solve| $\Omega_i$ problem \eqref{eq:pluriModel}}
    \STATE{update residual, check tolerance, update $iter$}
    \STATE \verb|[Uiprec, Viprec]| $\gets$  \verb|[Ui, Vi]|
    \ENDFOR
    \STATE \verb|[U0i, V0i]| $\gets$  \verb|[Ui, Vi]|
    \ENDWHILE
    \end{algorithmic}
\end{algorithm}

\subsection{Convergence check (Simona+Nicola+Teresa+Daniele)}
\subsection{Reliability (Simona+Nicola+Teresa+Daniele)}
\subsection{Eventuali sensitivity (Simona+Nicola+Teresa+Daniele)}

\section{DD per il modello completo (Simona+Nicola+Teresa+Daniele)}

\tb{si intende solo i risultati o anche la discretizzazione?}

\section{Conclusions and future developments (Simona+Daniele)}

\textbf{Conclusions and future developments}%
In this thesis we developed a multi-cellular model accounting for a spatially-extended intra-cellular system for ROPs pattern formation. In the proposed framework, we solved ROPs pattern formation in a system composed by multiple cells, together with a transport model for hormone auxin. The RD model explains the auxin-mediated action of ROPs in an Arabidopsis root hair cell leading to formation of the localized patches of activated ROPs.

Our simulations support conclusions reached in other works regarding ROPs dynamics under a-priori defined auxin distribution \cite{phdthesis:victor, intra1_R, intra2}. We have analyzed various scenarios such that a stripe-like patch forms where auxin concentration is higher. Then, instability of stripe into spot-like states occurs and multiple spots align with auxin gradient or travel towards auxin minimum. Several results confirm that, for a transversally independent gradient, lateral stripes become unstable states.

Successively as a new contribution, we extended the intra-cellular dynamics for root-hair initiation model to a multi-cellular system, developing a new model taking into account communication between cells. In order to do so, we have defined a boundary value problem which assumes new boundary conditions between neighboring cells. Subsequently, we develop an iterative procedure to solve it. We take as reference scheme a Robin-Robin domain decomposition method. We impose fluxes of ROPs betweeen neighboring cells, depending on the difference of the ROPs concentrations, through localised open channels. Such connections are tuned in order to visualize considerably different results with respect to a configuration characterized by stagnant cells. In addition,we aim at preserving all previous analyses over important parameters characterizing the system. We numerically assessed the robustness of the proposed model in cooperating with auxin distribution in influencing ROPs pattern formation.

Having defined a reliable multi-cellular model, we have taken into account also auxin concentration dynamics. Hormone auxin is regulated by carriers PIN following a non-linear ODEs system. We implement a semi-implicit method to solve such as a system on a two cells setting. The simulations we carried out show oscillating values of auxin concentration for specific sets of parameters, as expected from previous studies. A further confirmation of the robustness of pattern formation according to the proposed new multi-cellular model, when considering channel communication between cells, is given in Chapter \ref{cap:4}. It is shown that even if the system is under steady homogeneous auxin concentrations, the model robustly forms hotspots when considering open channels under a sufficiently high overall auxin level.

The results show two ways spots of active ROPs can be generated. The first factor is the auxin gradient, which still guarantees and influence stripe to spot evolution. The second factor is the structural coupling between cells. Even if not characterized by variation in space but with a sufficiently high value of the auxin concentrations, the multi-cellular model leads the system to multiple spots.

In conclusion, this thesis provides a first attempt in modeling communication between root-hair cells in pattern formation. Future developments include to study the structural model or to deeply analyze channels characterization. Moreover, the iterative procedure implemented could be applied to a multi-cellular system composed by more than four cells and eventually try to increase the computational performance through parallel computing. Another possible road to follow is to approach the problem through homogenization techniques. The cited method may increase the efficiency in solving the model, particularly when the number of cells involved becomes large. Finally, one could assume other auxin transport models, less simplified or accounting for spatial dependence inside the cells.





\bibliographystyle{siamplain}
\bibliography{references}
\end{document}
